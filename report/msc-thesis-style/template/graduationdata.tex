%De aankondiging bevat de spreker, titel, plaats, datum en tijd, samenstelling van de afstudeercommissie en een korte samenvatting (maximaal 25 regels).
\thispagestyle{empty}

\noindent \textbf{Pranav Sailesh Mani}\\
\begin{tabular}{l}
\reportAuthor{} (\reportUrlEmail)\\
\end{tabular}\\
\noindent \textbf{Title}\\
\begin{tabular}{l}
\reportTitle\\
\end{tabular}\\
\noindent \textbf{MSc presentation}\\
\begin{tabular}{l}
% <MM> DD, YYYY (like \today)
30 November, 2017\\
\end{tabular}

\vspace{1.1cm}

\noindent \textbf{Graduation Committee}\\
\begin{tabular}{ll}
\graduationCommittee
\end{tabular}
\vspace{1.1cm}

\noindent \begin{tabular}{ll}
\textbf{Thesis Number - CE-MS-2017-14}
\end{tabular}

\begin{abstract} %de abstract bevat alleen een korte samenvatting van de inhoud van het onderzoek
\setcounter{page}{3}
\reportAbstract{ Cameras are used in various applications and one such common application has been space. Delfi Space is the CubeSat Development program of the Delft University of Technology and a subprogram of Defi-Space is Delfi-PQ which aims to develop PocketQubes which are an order of magnitude smaller than the CubeSat standards. One of the advanced payloads that would potentially be a part of the Delfi-PQ is an imager/camera. The imager needs to be as small as possible in order to fit into the Delfi-PQ satellites. The design of the camera has remained the same throughout the years and one of the reasons that increase the thickness of the camera is the presence of lenses. A way to reduce the size of the camera would be to remove the lens out of the equation. However, this introduces additional problems and trade-offs in the camera. These lenses can be replaced by masks/coded apertures. One of the additional steps in using coded-apertures is that additional computational steps need to be performed in order to reconstruct the image. The kind of computation that needs to be performed depends on the kind of masks that would be used. In this thesis, a separable mask is chosen and computer simulations on separable masks have been performed. Two image sensors that can be used in the picosatellite were chosen for implementation and the hardware/software is designed and developed. The experimental setup for determining the field of view of a lensless camera has been developed and tested. One of the trade-offs observed through the experiments is that the acceptance angle of a lensless imager had reduced by 38 percent and 31.8 percent in the horizontal and vertical directions compared to a conventional lens-based system. Based on the experimental results, the field of view of the camera has also been determined. A singular value decomposition based method has been developed and is used to align and calibrate the camera with the mask. The final step is estimating the system matrices of the lensless system. The system matrices enable the perfect reconstruction and a complete realization of a separable mask based lensless camera.  A scheme for estimating the system matrices of the lensless imager using Hadamard basis is proposed and confirmed using simulations and the strategy for experimental verification is proposed. As far as we know, this is the first study that focuses on designing and developing a lensless camera in the visible light domain for use in picosatellites. }
\end{abstract}

\clearpage
