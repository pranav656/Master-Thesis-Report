\chapter{System Modelling and Design}
This chapter will describe how the system can be mathematically modelled and how the mask for the lensless imager was designed. As mentioned in the previous chapter the system can be modelled as 
\begin{equation}
\label{eq:conv2}
y = \phi * x + e ;
\end{equation}
Ignoring the noise and converting the equation to fourier domain, the equation \ref{eq:conv2} can be re-written as
\begin{equation}
\label{eq:conv3}
F(y) = F(\phi)F(x)
\end{equation}
\begin{equation}
\label{eq:conv3}
F(x) = \frac{F(y)}{F(\phi)}
\end{equation}
\begin{equation}
\label{eq:conv4}
x = F^{-1}(\frac{F(y)}{F(\phi)})
\end{equation}
Equation \ref{eq:conv4} is the simplest possible computational inversion of the scene from the sensor. This method has also been used in \cite{Toeplitz}. As mentioned in the previous chapter, there are two types of masks that can be used for the purpose of encoding the scene onto the mask, namely separable and non-separable mask. MATLAB has been used for the purpose of simulating the algorithms. 
\section{Simulation of a non-separable mask}

\section{Simulation of a separable mask}