\chapter{Conclusions and Future Work}
\label{chp:conclusionsandfuturework}

\section{Goals and Research Questions revisited}
In the beginning of the chapter, a research question and goals for the project were proposed. Now, let us revisit them to find out whether we have reached the goals of the project. The project was divided into these four sub-questions as given below:
\begin{itemize}
\item What would be the CMOS sensor that can be used for the camera? 
In chapter 2 of this report, the survey of CMOS sensors that have been used in previous CubeSat missions has been conducted. Based on various factors, OV2640 and OV5642 were chosen. This also indicates the achievement of the first goal.
\item How do we design the hardware and software for such a camera that can be used in Delfi-PQ satellite?\\
The main factor in choosing OV2640 and OV5642 was the presence of open source libraries,  and open source electronic interface hardware. The software and the hardware mechanism for controlling the cameras have been described in Chapter 4 of the report. All the experiments described in the later chapters of the report were developed using the same hardware and software described in Chapter 4. This also indicates the achievement of the second goal.
\item What would be the field-of-view and spatial resolution of the lensless camera?\\
An experimental setup to determine the acceptance cone of the CMOS sensor has been devised. The experimental results were incorporated into the previously completed simulations. Based on the experimental results, OV2640 had an acceptance angle of 43.6 degrees. It was found that the field of view of the lensless system had reduced 38 percent and 31.8 percent in the horizontal and vertical directions compared to a conventional lens-based system. The effective area of the sensor that could be used also reduced by 52 percent. The spatial resolution of the designed camera was calculated and the entire procedure for calculating the spatial resolution is described in Chapter 2 and Chapter 5 of the report. This also indicates the achievement of the third goal.
\item What would be the computational algorithm that would be used in such a lens-less camera? How do we experimentally prove the concept of lens-less imaging?\\
The computational algorithm mainly depends on the mask that we use for imaging. Two kinds of masks were evaluated namely, separable and non-separable masks. It was found that non-separable masks cannot be used to image objects in the visible light spectrum using existing computational methods available in the literature. So, we decided to go a separable Toeplitz mask. It was found in the simulations that separable Doubly-Toeplitz mask could be used to reconstruct objects even in the presence of diffraction effects. The entire simulation workflow is described in Chapter 3 of the report.

Experimental verification of lensless imaging required multiple stages of experiments. After determining the field of view of the OV2640 sensor, the imaging of the separable mask was tried using OV2640. However, it was found out that OV2640 could not be used due to the limitation of the onboard FIFO buffer of the camera. Because of this, we decided to use OV5642 which has a bigger FIFO buffer. OV5642 is able to image the mask perfectly and preserves the separable property of the mask. The entire experimental approach was done using singular value decomposition and is mentioned in Chapter 6 of the report.  

The next step was to determine the system matrices of the lensless camera. To do this, we decide to use a Hadamard basis matrix as it could provide enough amount of light to produce a measurable sensor response. Using a basis matrix also provided better reconstruction and preserved the original object property as seen in the simulation results. This method requires us to use $N \times N$ calibration patterns on the LCD if we want to reconstruct images of resolution $N \times N$. This method was not verified experimentally and forms the final step of proving the concept of lensless imaging experimentally. The strategy and scheme for achieving this are described in Chapter 7 of the report. This indicates that there is some more experimental verification required to achieve the final goal.

\end{itemize}

Now, let us come to the main research question:
\textbf{Is it possible to design ``lensless coded" aperture cameras with a small form-factor(thickness $<$ 10mm) using COTS(commercial off-the-shelf) components that can be used in U-class Spacecraft ?}\\
The experimental results with commercially available camera modules OV2640 and OV5642 provide a good insight into the lensless imaging methodologies. The previous studies done in this field did not have any memory limitations like we faced with OV2640. The camera with bigger memory such as OV5642 could retain the properties of the mask when we experimentally tested them. A separable scene on the outside also yields a separable scene on the CMOS sensor. A scheme for determining the system matrices of a lensless imaging system is designed and simulated. However, the experimental determination of this scheme has not yet been verified. The simulations and experiments done in this work provide a good picture and insight into the concept of lensless imaging as a whole. The experimental results achieved till now indicate that commercially available cameras can be used but more experiments need to be performed before this question can be answered with more certainty. 

\section{Future Work}
There is a great scope for developing the work mentioned in this thesis. This work can be improved and extended in the following ways:
\begin{itemize}
\item Experimental determination of System Matrices: A scheme for experimental determination of the system matrix has been described in Chapter 7 of the report. This needs to be completed to completely realize the concept of lensless imaging. The scheme described is extremely long as it requires $N \times N$ measurements to perform complete reconstructions. This scheme can also be improved to reduce the number of measurements needed to accurately estimate the system matrices. 

\item Fabrication with masks: This work uses transmissive Spatial light modulators(SLM) to simulate the effects of a mask. It was observed that a complete binary mask cannot be achieved with this equipment. Lithographic photomasks can be fabricated and can offer better performance than using an SLM.

\item Improving the computational method for reconstruction: This work uses regularization along with inversion to perform reconstructions. This method can also be improved using other methods to denoise the reconstruction such as SVD, BM3D and total variation based methods\cite{Flatcam}. These methods could be tried out on the existing simulations and studied whether they provide better reconstructions.

\item Hardware Improvements: OV2640 could not be used due to the limited onboard memory. The future work can also look into improving the camera hardware by replacing the memory of the camera module. This would also require a change in the hardware/software implementations of the library. Arducam can be contacted on how to achieve this. Also, they release improved versions of the camera modules every year.
\end{itemize}

