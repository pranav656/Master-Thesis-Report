\chapter{Introduction and Problem Statement}
\label{chp:introduction}
The history of cameras goes back to 13th century when Aristotle first noticed how light passing through a small hole in a darkened room produced an image of the sun on the wall. 
Throughout the centuries, the basic design of cameras has been continuously changing with different versions of the 'camera obscura'. 'Camera Obscura' is a phenomenon that occurs when a scene is projected onto a pinhole and the image of that scene is formed on the surface opposite to that of a pinhole. 

In a pinhole camera, light passes through the pinhole and forms an image on the sensor/image plane. As the size of the pinhole increased, the quality of the image formed on the plane decreased and as the pinhole size became smaller, the lesser light was allowed which resulted in a decreased field of view. With the development of science and due to the limitations of the pinhole, lenses were introduced to increase the size of the aperture, the sharpness of the image and the light throughput. As humanity progressed with the rapid pace of technology, we were able to capture images and store them in a film. With the digital explosion in the early 1990s, the thin films were replaced by Charged Couple Devices(CCD). Then came the cameras based on Complementary Metal Oxide Semiconductors(CMOS). CCD and CMOS sensors reduced the size of cameras considerably and it was possible to develop low-cost cameras in a large number. However, cameras have retained the lens throughout the years. Cameras are used for various applications and one such application is the space exploration domain. 

\section{Context}
The Delfi program focuses on the development of CubeSats at the Delft University of Technology, to provide hands-on education for students and also provide technology demonstrations for the Dutch Space Industry. The satellites that have been launched under the Delfi program are as follows:
\begin{itemize}
\item Delfi-C3: Delfi-C3 is the first university class satellite and also the first nano-satellite from the Netherlands. It was launched on April 28, 2008, from India. Delfi-C3 is a full mission success and is still operational\cite{DelfiC3}.
\item Delfi-n3Xt: The Delfi-n3Xt(pronounced as Delfi-Next) is the second set of CubeSats developed as the successor to Delfi-C3. It was launched from Russia on November 21, 2013. It was able to log data for three months and perform all foreseen technology demonstration experiments.
\end{itemize}

Delfi-PQ programme is a sub-programme of the Delfi Space programme that aims at developing extremely small but highly capable PocketQube satellites. PocketQubes are an order of magnitude smaller than the well known CubeSat standard which formed the basis of previous Delfi satellite projects. The dimensions of a PocketQube satellite would be 50mm * 50mm * 178mm and their volume would be approximately eight times smaller than CubeSats. One of the advanced payloads that would be part of the Delfi-PQ would be an imager/camera that consumes extremely low power and would fit into the dimensions specified by the Delfi-PQ team. The thickness of the camera should be less than 10 mm thick.
 
 
\section{Problem Statement}
In order to reduce the size of a camera, it would be necessary to remove the lens from the camera as the thinnest lens based mobile camera is 5mm thick. The primary focus of a lens would be to focus light from distant objects onto the CMOS sensor. Light from distant objects reaches the sensor even without the lens except that the light is incoherent and the CMOS sensor would not be able to form the object properly without a lens. However, the lens could also be replaced by coded apertures. Coded Apertures have been used in the late 20th century to image X-Ray sources of light. Lensless coded aperture cameras can be as small as $500 \mu m$ thick. By using lensless cameras, we could potentially reduce the form-factor multiple times to suit the requirements of Delfi-PQ. Apart from this, it would be the first attempt to use a lens-less camera for use in satellites to image astronomical objects in the visible light spectrum. In the case of lensed imaging, the image of the scene is directly obtained on the sensor. That does not happen in the case of lens-less imaging. In lensless imaging, one would need to computationally reconstruct the object scene using various computational methods. This is one of the trade-offs that we need to sacrifice in the case of lensless imaging.
This thesis would address the following research question:
\textbf{Is it possible to design ``lensless coded" aperture cameras with a small form-factor(thickness $<$ 10mm) using COTS(commercial off-the-shelf) components that can be used in U-class Spacecraft ?}
%However, the thesis would focus on the broader applications of a lensless camera in satellites and would also make an attempt at addressing the camera satisfying the size requirements of the Delfi-PQ.

This question can be broken down into the following sub-questions:
\begin{itemize}
\item What would be the CMOS sensor that can be used for the camera? 
\item How do we design the hardware and software for such a camera that can be used in Delfi-PQ satellite?
\item What would be the field-of-view and spatial resolution of the lensless camera?
\item What would be the computational algorithm that would be used in such a lens-less camera? How do we experimentally prove the concept of lens-less imaging?

%\item Would it be possible to design a lensless camera to capture astronomical objects in the visible range of light spectrum?
%\item What would be the minimum possible form-factor that would be achievable and the effects of different factors such as diffraction effects, mask-to-sensor distance, and reconstruction algorithms?
%\item If possible, how would the lensless camera compare with the conventional lens-based cameras used currently?
%\item Would it be possible to design a lensless camera that would fit the size constraints of the Delfi-PQ?
\end{itemize}

Based on the above set of questions, the following goals have been determined for the project:
\begin{itemize}
\item Perform a survey of previously used CMOS sensors in various CubeSat missions around the world. Choose a CMOS sensor that could be used for lensless imaging.
\item Design the hardware and software prototype using a commercially available micro-controller platform.
\item Design an experimental setup for determining the field-of-view and spatial resolution of the lens-less camera.
\item Design an experimental setup to prove the concept of lensless imaging using existing computational methods. 
\end{itemize}
\section{Assumptions}
The thesis makes the following assumptions:
\begin{itemize}
\item The imaging system needs to use a COTS micro-controller platform that could be easily integrated with onboard computer module on the Delfi-PQ. This assumption is fair as the previous Delfi satellites use commercially available micro-controller platforms for use.
\item The second assumption is that it would be possible to stream down the data from the satellite to the earth and perform computations in the ground station and perform computational reconstructions. Lens-less Imaging requires computational reconstruction and it would be a computationally complex task to perform complex Fourier computations on-board satellite. Apart from this, various computational techniques can be added to perform better reconstructions at a later stage. This assumption is valid since the previous Delfi satellites periodically send back data to the earth. This assumption allows a greater flexibility while designing the system.
\item It is also assumed that there is no loss of image data in the transmission between the satellite and ground-station and that the image is sent as stored by the imaging module.
\end{itemize}
\begin{figure}[htb]
\includegraphics[width=\textwidth]{pics/Sys-Arch-1}
\caption{Assumed System Architecture}
\label{fig:System Architecture}
\end{figure}

\vspace{1\baselineskip}

\noindent


