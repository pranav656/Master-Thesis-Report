\chapter{Introduction and Problem Statement}
\label{chp:introduction}
The history of cameras go back to 13th century when Aristotle first noticed how light passing through a small hole in a darkened room produced an image of the sun on the wall. 
Throughout the centuries, the basic design of cameras have been continuously changing with  different versions of the 'camera obscura'. 'Camera Obscura' is a phenomenon that occurs when when a scene is projected onto a pinhole and the image of that scene is formed on the surface opposite to that of a pinhole. 

In a pinhole camera, light passes through the pinhole and forms an image on the sensor/image plane.As the size of the pinhole increased, the quality of image formed on the plane decreased and as the pinhole size became smaller, lesser light was allowed which resulted in decreased field of view. With the development of science and due to the limitations of the pinhole, lenses were introduced to increase the size of the aperture, the sharpness of the image and the light throughput. As humanity progressed with the rapid pace in technology, we were able to capture images and store them on a film. With the digital explosion in early 1990s, the thin films were replaced by Charged Couple Devices(CCD). Then came the cameras based on Complementary Metal Oxide Semiconductors(CMOS). CCD and CMOS sensors reduced the size of cameras considerably and it was possible to develop low cost cameras in a large number. However, cameras have retained the lens throughout the years. Cameras are used for various applications and one such application is the space exploration domain. 

Delfi Space is the small satellite program of TU Delft that is mainly meant for education and technology demonstration in very small sized satellites. Delfi-PQ programme is a sub-programme of the Delfi Space programme that aims at developing extremely small but highly capable PocketQube satellites. PocketQubes are an order of magnitude smaller than the well known CubeSat standard which formed the basis of previous Delfi satellite projects. The dimensions of a PocketQube satellite would be 50mm * 50mm * 178mm and their volume would be approximately eight times smaller than CubeSats. One of the advanced payload that would be part of the Delfi-PQ would be an imager/camera that consumes extremely low power and would fit into the dimensions specified by the Delfi-PQ team. The thickness of the camera should be less than 6mm thick.
 
In order to reduce the size of a camera, it would be necessary to remove the lens from the camera as the thinnest lens based mobile camera is 5mm thick. The primary focus of a lens would be to focus light from distant objects onto the CMOS sensor. Light from distant objects reach the sensor even without the lens except that the light is incoherent and the CMOS sensor would not be able to form the object properly without a lens. However, the lens could also be replaced by coded apertures. Coded Apertures have been used in the late 20th century to image X-Ray sources of light. Lensless coded aperture cameras can be as small as $100 \mu m$ thick. By using lensless cameras, we could potentially reduce the form-facto multiple times to suit the requirements of Delfi-PQ. This thesis would address the following research question:


\textbf{Is it possible to design ``lensless coded" aperture cameras with a small form-factor(thickness $<$ 10mm) using COTS(commercial off-the-shelf) components that can be used in U-class Spacecrafts ?}
%However, the thesis would focus on the broader applications of lensless camera in satellites and would also make an attempt at addressing the camera satisfying the size requirements of the Delfi-PQ.

This question can be broken down into the following sub-questions:
\begin{itemize}
\item Would it be possible to design a lensless camera to capture astronomical objects in the visible range of light spectrum?
\item What would be the minimum possible form-factor that would be achievable and the effects of different factors such as diffraction effects, mask-to-sensor distance and reconstruction algorithms?
\item If possible, how would the lensless camera compare with the conventional lens based cameras used currently?
\item Would it be possible to design a lensless camera that would fit the size constraints of the Delfi-PQ?
\end{itemize}
\vspace{1\baselineskip}

\noindent


